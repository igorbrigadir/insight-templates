% !TEX root = ../thesis.tex
%%% Local Variables: 
%%% mode: latex
%%% TeX-master: "thesis"
%%% End: 

\chapter{Chapter Title}\label{chap:chapter_name}

Chapter introduction goes here.

\section{Section heading}\label{section:section_name}

Section description goes here.

And if I want to include a graphic, I can use the code below and reference it \ref{graphics:sample_graphic}.

\begin{figure}\label{graphics:sample_graphic}	
	\centering
		\includegraphics[width=\textwidth]{graphics/logos/SFI.png}
	\caption{Sample image for display}
	\label{graphics:sample_graphic}
\end{figure}

Although, figures, tables and equations will not necessarily appear where they are in the .tex file .. 

\subsection{Subsection Heading}\label{subsection:subsection_name}

Subsection description goes here. 

\subsubsection{Subsubsection Heading}\label{subsubsection:subsubsection_name}

Subsubsection description goes here. And if needed, I can include paragraphs, like so. 

\paragraph{Paragraph heading}

Paragraph text (if needed). I can also include equations, and reference them \ref{equation:tiling_operation}.

% Comment: Change this to allow overlapping windows 
\begin{equation}\label{equation:tiling_operation}
	\alpha \cap \omega
\end{equation}

And I can also use tables, and reference them \ref{table:sample_table}.

\begin{table}[]
	\begin{center}
		\begin{tabular}{r r } \hline \hline
			Column 1 & Column 2 \\
			\hline			
			A & B \\
			C & E \\
			D & F \\
		\end{tabular}
	\end{center}
	\caption{Sample table}
	\label{table:sample_table}
\end{table}


\subsection{Subsection Heading 2}\label{subsection:subsection_name_2}

And so on ... 